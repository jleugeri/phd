
\colorlet{highlightColor_binary}{myBlue}
\colorlet{highlightColor_digital}{myPurple}
\colorlet{highlightColor_continuous}{myGreen}
\colorlet{highlightColor_sigmadelta}{myMagenta}
\colorlet{highlightColor_spiking}{myOrange}
\colorlet{highlightColor_filter}{myBlue}

\tikzset{
    node distance=5mm,>=latex,
    pulse/.style={
        minimum size=(1.5*#1),
        append after command={ (\tikzlastnode.center) ++(-0.5*#1,-0.5*#1) -- ++(0.25*#1, 0pt) -- ++(0pt, #1) -- ++(0.5*#1, 0pt) -- ++(0pt,-#1) -- ++(0.25*#1,0pt)},
    },
    relu/.style={
        minimum size=(1.5*#1),
        append after command={ (\tikzlastnode.center) ++(-0.5*#1,-0.5*#1) -- ++(0.5*#1, 0pt) -- ++(0.5*#1, #1)},
    },
    linear/.style={
        minimum size=(1.5*#1),
        append after command={ (\tikzlastnode.center) ++(-0.5*#1,-0.5*#1) -- ++(#1, #1)},
    },
    quadratic/.style={
        minimum size=(1.5*#1),
        append after command={ (\tikzlastnode.center) ++(-0.5*#1,-0.5*#1) +(0,#1) parabola[parabola height=-#1] +(#1, #1)},
    },
    exponential/.style={
        minimum size=(1.5*#1),
        append after command={ (\tikzlastnode.center) ++(-0.5*#1,-0.5*#1) .. controls +(0.5*#1,0) .. ++(#1,#1)},
    },
    step/.style={
        minimum size=(1.5*#1),
        append after command={ (\tikzlastnode.center) ++(-0.5*#1,-0.5*#1) -- ++(0.5*#1, 0pt) -- ++(0pt, #1) -- ++(0.5*#1, 0pt)},
    },
    stair/.style={
        minimum size=(1.5*#1),
        append after command={ (\tikzlastnode.center) ++(-0.5*#1,-0.5*#1) -- ++(0.5*#1, 0pt) -- ++(0pt, 0.25*#1) -- ++(0.125*#1, 0pt) -- ++(0pt, 0.25*#1) -- ++(0.125*#1, 0pt) -- ++(0pt, 0.25*#1) -- ++(0.125*#1, 0pt) -- ++(0pt, 0.25*#1) -- ++(0.125*#1, 0pt) },
    },
    digital/.style={
        double,
        >=angle 90,
    },
    binary/.style={
    },
    continuous/.style={
        decorate,decoration={snake, segment length=3pt, amplitude=1pt, pre length=4pt, post length=4pt}
    },
    pulsed/.style={
        densely dashdotted
    },
    weight/.style={
        draw, rectangle, rounded corners
    },
    operation/.style={
        draw, circle
    },
    block/.style={
        draw, rectangle, rounded corners
    },
    hidden/.style={
        opacity=0.25
    },
    highlight/.style args={#1}{
        left color=#1!30!white,
        right color=#1!30!white,
        fill=#1!30!white, text=#1, rounded corners
    }
}
    
% draw digital neuron
\makeatletter
\newcommand{\drawDigitalNeuronInner}[3][]{
    %draw first neuron
    \node[operation, label=below:{$x[t]$}]  (sum) at (0,0) {+};
    \draw node[block, stair=12pt] (nonlinearity) at ($(sum)+(1,0)$) {};
    \coordinate (transmitter) at ($(nonlinearity)+(4em,0)$) ;
    \node[weight, label={below:{$w_{1}$}}] (weight1) at ($(sum)+(-1,1.1)$) {$\times$};
    \node[weight, label={below:{$w_{2}$}}]  (weight2) at ($(sum)+(-1,0)$) {$\times$};
    \node[] (dots) at ($(sum)+(-1,-1.1)$) {\vdots};

    % draw signals
    \ifnum\numexpr#3\relax>0{
        \draw[<-,decorate,digital] (weight1) -- node[below=0.2]  (input1) {$s_1[t]$} ++(-4em,0);
        \draw[<-,decorate,digital] (weight2) -- node[below=0.2]  (input2) {$s_2[t]$} ++(-4em,0);
    }\fi;

    \draw [->] (weight1) -| (sum);
    \draw [->] (weight2) -- (sum);
    \draw [] (sum) -- (nonlinearity);
    
    % draw background
    \begin{scope}[on background layer]
        \path [highlight=highlightColor_digital] let \p1=(weight1.north west), \p2=(dots.south west), \p3=(nonlinearity.east) in 
        ($(\x1,\y1) + (-0.1,0.1)$) -- node[above]{#2} ($(\x3,\y1) + (0.1,0.1)$) -- ($(\x3,\y2)+(0.1,-0.1)$) -- ($(\x1,\y2)+(-0.1,-0.1)$) -- cycle;
    \end{scope}
    
    
    \ifnum\numexpr#3\relax=1{
        \draw[->,decorate,digital] (nonlinearity) -- node[below=0.2]  (input2) {$y[t]$} (transmitter);
    } \else \ifnum\numexpr#3\relax>1 {
        %draw second neuron
        \path (transmitter) ++(1.5,+1) node[weight, hidden, label={below:{$w_{3}$}}] (weight3) {$\times$}
        ++(0,-1) node[weight, label={below:{$w_{4}$}}]  (weight4) {$\times$}
        ++(0,-1) node[]  (dots2) {\vdots};
        \node[operation, hidden, right=of weight4, label={below:$z[t]$}]  (sum2) {+};
        \node[right=of sum2] {...};

        % draw background
        \begin{scope}[on background layer]
                \path [highlight=highlightColor_digital, right color=white] let \p1=(weight3.north west), \p2=(dots2.south west), \p3=(weight3.west), \p4=(sum2.east) in 
            ($(\x3,\y1) + (-0.1,0.1)$) -- node[above]{#1} ($(\x4,\y1) + (0.2,0.1)$) -- ($(\x4,\y2)+(0.2,-0.1)$) -- ($(\x3,\y2)+(-0.1,-0.1)$) -- cycle;
        \end{scope}

        \draw[->,decorate,digital] (nonlinearity) -- node[below=0.2]  (input2) {$y[t]$} (weight4);
        \draw [->, hidden] (weight3) -| (sum2);
        \draw (weight4) -- (sum2);
    } \fi \fi;
}
\makeatother
\newcommand{\drawDigitalNeuron}[3][1]{
    \begin{tikzpicture}
        \drawDigitalNeuronInner[#1]{#2}{#3};
    \end{tikzpicture}
}
    
% draw binary neuron
\makeatletter
\newcommand{\drawBinaryNeuronInner}[3][\@nil]{
    \def\optarg{#1}
    
    %draw first neuron
    \node[operation, label=below:{$x[t]$}]  (sum) at (0,0) {+};
    \draw node[block, step=12pt] (nonlinearity) at ($(sum)+(1,0)$) {};
    \coordinate (transmitter) at ($(nonlinearity)+(4em,0)$) ;
    \node[weight, label={below:{$w_{1}$}}] (weight1) at ($(sum)+(-1,1.1)$) {$\times$};
    \node[weight, label={below:{$w_{2}$}}]  (weight2) at ($(sum)+(-1,0)$) {$\times$};
    \node[] (dots) at ($(sum)+(-1,-1.1)$) {\vdots};

    % draw signals
    \ifnum\numexpr#3\relax>0{
        \draw[<-,decorate,binary] (weight1) -- node[below=0.2]  (input1) {$s_1[t]$} ++(-4em,0);
        \draw[<-,decorate,binary] (weight2) -- node[below=0.2]  (input2) {$s_2[t]$} ++(-4em,0);
    }\fi;

    \draw [->] (weight1) -| (sum);
    \draw [->] (weight2) -- (sum);
    \draw [] (sum) -- (nonlinearity);
    
    % draw background
    \begin{scope}[on background layer]
        \path [highlight=highlightColor_binary] let \p1=(weight1.north west), \p2=(dots.south west), \p3=(nonlinearity.east) in 
        ($(\x1,\y1) + (-0.1,0.1)$) -- node[above]{#2} ($(\x3,\y1) + (0.1,0.1)$) -- ($(\x3,\y2)+(0.1,-0.1)$) -- ($(\x1,\y2)+(-0.1,-0.1)$) -- cycle;
    \end{scope}
    
    
    \ifnum\numexpr#3\relax=1{
        \draw[->,decorate,binary] (nonlinearity) -- node[below=0.2]  (input2) {$y[t]$} (transmitter);
    } \else \ifnum\numexpr#3\relax>1 {
        %draw second neuron
        \path (transmitter) ++(1.5,+1) node[weight, hidden, label={below:{$w_{3}$}}] (weight3) {$\times$}
        ++(0,-1) node[weight, label={below:{$w_{4}$}}]  (weight4) {$\times$}
        ++(0,-1) node[]  (dots2) {\vdots};
        \node[operation, hidden, right=of weight4, label={below:$z[t]$}]  (sum2) {+};
        \node[right=of sum2] {...};

        % draw background
        \begin{scope}[on background layer]
                \path [highlight=highlightColor_binary, right color=white] let \p1=(weight3.north west), \p2=(dots2.south west), \p3=(weight3.west), \p4=(sum2.east) in 
            ($(\x3,\y1) + (-0.1,0.1)$) -- node[above]{#1} ($(\x4,\y1) + (0.2,0.1)$) -- ($(\x4,\y2)+(0.2,-0.1)$) -- ($(\x3,\y2)+(-0.1,-0.1)$) -- cycle;
        \end{scope}

        \draw[->,decorate,binary] (nonlinearity) -- node[below=0.2]  (input2) {$y[t]$} (weight4);
        \draw [->, hidden] (weight3) -| (sum2);
        \draw (weight4) -- (sum2);
    } \fi \fi;

}
\makeatother
\newcommand{\drawBinaryNeuron}[3][1]{
    \begin{tikzpicture}
        \drawBinaryNeuronInner[#1]{#2}{#3};
    \end{tikzpicture}
}

% draw linear nonlinear neuron
\makeatletter
\newcommand{\drawContinuousNeuronInner}[3][\@nil]{
    \def\optarg{#1}
    \def\drawarrow{#3}
    %draw first neuron
    \node[operation, label=below:{$x(t)$}]  (sum) at (0,0) {+};
    \draw node[block, relu=12pt] (nonlinearity) at ($(sum)+(1,0)$) {};
    \coordinate (transmitter) at ($(nonlinearity)+(4em,0)$) ;
    \node[weight, label={below:{$w_{1}$}}] (weight1) at ($(sum)+(-1,1.1)$) {$\times$};
    \node[weight, label={below:{$w_{2}$}}]  (weight2) at ($(sum)+(-1,0)$) {$\times$};
    \node[] (dots) at ($(sum)+(-1,-1.1)$) {\vdots};

    % draw signals
    \ifnum\numexpr#3\relax=1{
        \draw[<-,decorate,continuous] (weight1) -- node[below=0.2]  (input1) {$s_1(t)$} ++(-4em,0);
        \draw[<-,decorate,continuous] (weight2) -- node[below=0.2]  (input2) {$s_2(t)$} ++(-4em,0);
    } \fi;
    
    \draw [->] (weight1) -| (sum);
    \draw [->] (weight2) -- (sum);
    \draw [] (sum) -- (nonlinearity);
    
    % draw background
    \begin{scope}[on background layer]
        \path [highlight=highlightColor_continuous] let \p1=(weight1.north west), \p2=(dots.south west), \p3=(nonlinearity.east) in 
        ($(\x1,\y1) + (-0.1,0.1)$) -- node[above]{#2} ($(\x3,\y1) + (0.1,0.1)$) -- ($(\x3,\y2)+(0.1,-0.1)$) -- ($(\x1,\y2)+(-0.1,-0.1)$) -- cycle;
    \end{scope}
    
    
    \ifnum\numexpr#3\relax=1{
        \draw[->,decorate,continuous] (nonlinearity) -- node[below=0.2]  (input2) {$y(t)$} (transmitter);
    } \else \ifnum\numexpr#3\relax>1 {            %draw second neuron
        \path (transmitter) ++(1.5,+1) node[weight, hidden, label={below:{$w_{3}$}}] (weight3) {$\times$}
        ++(0,-1) node[weight, label={below:{$w_{4}$}}]  (weight4) {$\times$}
        ++(0,-1) node[]  (dots2) {\vdots};
        \node[operation, hidden, right=of weight4, label={below:$z(t)$}]  (sum2) {+};
        \node[right=of sum2] {...};

        % draw background
        \begin{scope}[on background layer]
                \path [highlight=highlightColor_continuous, right color=white] let \p1=(weight3.north west), \p2=(dots2.south west), \p3=(weight3.west), \p4=(sum2.east) in 
            ($(\x3,\y1) + (-0.1,0.1)$) -- node[above]{#1} ($(\x4,\y1) + (0.2,0.1)$) -- ($(\x4,\y2)+(0.2,-0.1)$) -- ($(\x3,\y2)+(-0.1,-0.1)$) -- cycle;
        \end{scope}

        \draw[->,decorate,continuous] (nonlinearity) -- node[below=0.2]  (input2) {$y(t)$} (weight4);
        \draw [->, hidden] (weight3) -| (sum2);
        \draw (weight4) -- (sum2);
    } \fi \fi;
}
\makeatother
\newcommand{\drawContinuousNeuron}[3][1]{
    \begin{tikzpicture}
        \drawContinuousNeuronInner[#1]{#2}{#3};
    \end{tikzpicture}
}

% draw Sigma-Delta modulator
\makeatletter
\newcommand{\drawSigmaDeltaModulatorInner}[3][\@nil]{
    \def\optarg{#1}
    \def\drawarrow{#3}
    %draw first neuron
    \node[operation]  (sum) at (0,0) {+};
    \node[operation, label={above:$x[t]$}] (sub) at ($(sum)+(1,0)$) {+};
    \node[block]  (filter) at ($(sub)+(1,0)$) {$\frac{1}{z}$};
    \draw node[block, pulse=12pt] (nonlinearity) at ($(filter)+(1,0)$) {};
    \coordinate (transmitter) at ($(nonlinearity)+(0.1 + 4em,0)$) ;
    \node[weight, label={below:{$w_{1}$}}] (weight1) at ($(sum)+(-1,1.1)$) {$\times$};
    \node[weight, label={below:{$w_{2}$}}]  (weight2) at ($(sum)+(-1,0)$) {$\times$};
    \node[] (dots) at ($(sum)+(-1,-1.1)$) {\vdots};

    % draw signals
    \ifnum\numexpr#3\relax=1{
        \draw[<-,decorate,pulsed] (weight1) -- node[below=0.2]  (input1) {$s_1(t)$} ++(-4em,0);
        \draw[<-,decorate,pulsed] (weight2) -- node[below=0.2]  (input2) {$s_2(t)$} ++(-4em,0);
    } \fi;

    \draw [->] (weight1) -| (sum);
    \draw [->] (weight2) -- (sum);
    \draw [] (sum) -- (sub) -- (filter) -- (nonlinearity);
    \draw [->] (nonlinearity.east) -- ++(0.1,0) -- ++(0,-1) -| node[pos=0.1,above] {reset} node[near end,left]{$-\theta$} (sub);

    
    % draw background
    \begin{scope}[on background layer]
        \path [highlight=highlightColor_sigmadelta] let \p1=(weight1.north west), \p2=(dots.south west), \p3=(nonlinearity.east) in 
        ($(\x1,\y1) + (-0.1,0.1)$) -- node[above]{#2} ($(\x3,\y1) + (0.2,0.1)$) -- ($(\x3,\y2)+(0.2,-0.1)$) -- ($(\x1,\y2)+(-0.1,-0.1)$) -- cycle;
    \end{scope}
    
    
    \ifnum\numexpr#3\relax=1{
        \draw[->,decorate,pulsed] (nonlinearity) -- node[below=0.2]  (input2) {$y[t]$} (transmitter);
    } \else \ifnum\numexpr#3\relax>1 {            %draw second neuron
        \path (transmitter) ++(1.5,+1) node[weight, hidden, label={below:{$w_{3}$}}] (weight3) {$\times$}
        ++(0,-1) node[weight, label={below:{$w_{4}$}}]  (weight4) {$\times$}
        ++(0,-1) node[]  (dots2) {\vdots};
        \node[operation, hidden, right=of weight4, label={below:$z[t]$}]  (sum2) {+};
        \node[right=of sum2] {...};

        % draw background
        \begin{scope}[on background layer]
                \path [highlight=highlightColor_sigmadelta, right color=white] let \p1=(weight3.north west), \p2=(dots2.south west), \p3=(weight3.west), \p4=(sum2.east) in 
            ($(\x3,\y1) + (-0.1,0.1)$) -- node[above]{#1} ($(\x4,\y1) + (0.2,0.1)$) -- ($(\x4,\y2)+(0.2,-0.1)$) -- ($(\x3,\y2)+(-0.1,-0.1)$) -- cycle;
        \end{scope}

        \draw[->,decorate,pulsed] (nonlinearity) -- node[below=0.2]  (input2) {$y[t]$} (weight4);
        \draw [->, hidden] (weight3) -| (sum2);
        \draw (weight4) -- (sum2);
    } \fi \fi;
}
\makeatother
\newcommand{\drawSigmaDeltaModulator}[3][1]{
    \begin{tikzpicture}
        \drawSigmaDeltaModulatorInner[#1]{#2}{#3};
    \end{tikzpicture}
}

% draw spiking neuron
\makeatletter
\newcommand{\drawSpikingNeuronInner}[3][\@nil]{
    \def\optarg{#1}
    \def\drawarrow{#3}
    %draw first neuron
    \node[operation]  (sum) at (0,0) {+};
    \node[operation, label={above:$x(t)$}] (sub) at ($(sum)+(1,0)$) {+};
    \node[block]  (filter) at ($(sub)+(1,0)$) {$\frac{\alpha}{s+\alpha}$};
    \draw node[block, pulse=12pt] (nonlinearity) at ($(filter)+(1,0)$) {};
    \coordinate (transmitter) at ($(nonlinearity)+(0.1 + 4em,0)$) ;
    \node[weight, label={below:{$w_{1}$}}] (weight1) at ($(sum)+(-1,1.1)$) {$\times$};
    \node[weight, label={below:{$w_{2}$}}]  (weight2) at ($(sum)+(-1,0)$) {$\times$};
    \node[] (dots) at ($(sum)+(-1,-1.1)$) {\vdots};

    % draw signals
    \ifnum\numexpr#3\relax>0{
        \draw[<-,decorate,pulsed] (weight1) -- node[below=0.2]  (input1) {$s_1(t)$} ++(-4em,0);
        \draw[<-,decorate,pulsed] (weight2) -- node[below=0.2]  (input2) {$s_2(t)$} ++(-4em,0);
    } \fi;
    \draw [->] (weight1) -| (sum);
    \draw [->] (weight2) -- (sum);
    \draw [] (sum) -- (sub) -- (filter) -- (nonlinearity);
    \draw [->] (nonlinearity.east) -- ++(0.1,0) -- ++(0,-1) -| node[pos=0.1,above] {reset} node[near end,left]{$-\theta$} (sub);

    
    % draw background
    \begin{scope}[on background layer]
        \path [highlight=highlightColor_spiking] let \p1=(weight1.north west), \p2=(dots.south west), \p3=(nonlinearity.east) in 
        ($(\x1,\y1) + (-0.1,0.1)$) -- node[above]{#2} ($(\x3,\y1) + (0.2,0.1)$) -- ($(\x3,\y2)+(0.2,-0.1)$) -- ($(\x1,\y2)+(-0.1,-0.1)$) -- cycle;
    \end{scope}
    
    
    \ifnum\numexpr#3\relax=1{
        \draw[->,decorate,pulsed] (nonlinearity) -- node[below=0.2]  (input2) {$y(t)$} (transmitter);
    } \else \ifnum\numexpr#3\relax>1 {            %draw second neuron
        \path (transmitter) ++(1.5,+1) node[weight, hidden, label={below:{$w_{3}$}}] (weight3) {$\times$}
        ++(0,-1) node[weight, label={below:{$w_{4}$}}]  (weight4) {$\times$}
        ++(0,-1) node[]  (dots2) {\vdots};
        \node[operation, hidden, right=of weight4, label={below:$z(t)$}]  (sum2) {+};
        \node[right=of sum2] {...};

        % draw background
        \begin{scope}[on background layer]
                \path [highlight=highlightColor_spiking, right color=white] let \p1=(weight3.north west), \p2=(dots2.south west), \p3=(weight3.west), \p4=(sum2.east) in 
            ($(\x3,\y1) + (-0.1,0.1)$) -- node[above]{#1} ($(\x4,\y1) + (0.2,0.1)$) -- ($(\x4,\y2)+(0.2,-0.1)$) -- ($(\x3,\y2)+(-0.1,-0.1)$) -- cycle;
        \end{scope}

        \draw[->,decorate,pulsed] (nonlinearity) -- node[below=0.2]  (input2) {$y(t)$} (weight4);
        \draw [->, hidden] (weight3) -| (sum2);
        \draw (weight4) -- (sum2);
        } \fi \fi;
}
\makeatother
\newcommand{\drawSpikingNeuron}[3][1]{
    \begin{tikzpicture}
        \drawSpikingNeuronInner[#1]{#2}{#3};
    \end{tikzpicture}
}


% draw spiking single signal
\makeatletter
\newcommand{\drawSpikingNeuronSingleInputInner}[3][\@nil]{
    \def\optarg{#1}
    \def\drawarrow{#3}
    %draw first neuron
    \node[operation]  (sum) at (0,0) {+};
    \node[block]  (filter) at ($(sum)+(1,0)$) {$\frac{\alpha}{s+\alpha}$};
    \draw node[block, pulse=12pt] (nonlinearity) at ($(filter)+(1,0)$) {};
    \coordinate (transmitter) at ($(nonlinearity)+(0.1 + 4em,0)$) ;

    % draw signals
    \ifnum\numexpr#3\relax>0{
        \draw[<-,decorate,pulsed] (sum) -- node[below=0.2]  (input2) {$s(t)$} ++(-4em,0);
    } \fi;
    \draw [] (sum) -- (filter) -- (nonlinearity);
    \draw [->] (nonlinearity.east) -- ++(0.1,0) -- ++(0,-1) -| node[pos=0.1,above] {reset} node[near end,left]{$-\theta$} (sum);

    
    % draw background
    \begin{scope}[on background layer]
        \path [highlight=highlightColor_spiking] let \p1=($(sum.north west)+(-0.2,0.2)$), \p2=($(filter.south west)+(0,-0.75)$), \p3=(nonlinearity.east) in 
        ($(\x1,\y1) + (-0.1,0.1)$) -- node[above]{#2} ($(\x3,\y1) + (0.2,0.1)$) -- ($(\x3,\y2)+(0.2,-0.1)$) -- ($(\x1,\y2)+(-0.1,-0.1)$) -- cycle;
    \end{scope}
    
    
    \ifnum\numexpr#3\relax=1{
        \draw[->,decorate,pulsed] (nonlinearity) -- node[below=0.2]  (input2) {$y(t)$} (transmitter);
    } \else \ifnum\numexpr#3\relax>1 {            %draw second neuron
        \path (transmitter) ++(1.5,0) node[weight, label={below:{$\theta$}}]  (weight4) {$\times$};
        \node[block, label={below:$z(t)$}]  (filter2) at ($(weight4)+(1,0)$) {$\frac{\alpha}{s+\alpha}$};

        % draw background
        \begin{scope}[on background layer]
                \path [highlight=highlightColor_spiking, right color=white] let \p1=($(sum.north west)+(-0.2,0.2)$), \p2=($(filter2.south west)+(0,-0.75)$), \p3=(weight4.west), \p4=(filter2.east) in 
            ($(\x3,\y1) + (-0.1,0.1)$) -- node[above]{#1} ($(\x4,\y1) + (0.2,0.1)$) -- ($(\x4,\y2)+(0.2,-0.1)$) -- ($(\x3,\y2)+(-0.1,-0.1)$) -- cycle;
        \end{scope}

        \draw[->,decorate,pulsed] (nonlinearity) -- node[below=0.2]  (input2) {$y(t)$} (weight4);
        \draw (weight4) -- (filter2);
        } \fi \fi;
}
\makeatother
\newcommand{\drawSpikingNeuronSingleInput}[3][1]{
    \begin{tikzpicture}
        \drawSpikingNeuronSingleInputInner[#1]{#2}{#3};
    \end{tikzpicture}
}



% relationships
\newcommand{\drawRelationships}{
    \begin{tikzpicture}[
        relation/.style={
            >=stealth,
            line width=5pt,
            rounded corners,
            color=black!30!white,
        }
    ]
        % draw binary neuron
        \begin{scope}[local bounding box=binary_scope]
            \drawBinaryNeuronInner{binary neuron}{0};
        \end{scope}


        \begin{scope}[shift={(3.5,2)}]
            % draw digital neuron
            \begin{scope}[local bounding box=digital_scope]
                \drawDigitalNeuronInner{digital neuron}{0};
            \end{scope}

            % draw ΣΔ modulator
            \begin{scope}[shift={(4.5,0)},local bounding box=sigmadelta_scope]
                \drawSigmaDeltaModulatorInner{first-order $\Sigma\Delta$-modulator}{0};
            \end{scope}
        \end{scope}

        \begin{scope}[shift={(3.5,-2)}]
            % draw continuous LN neuron
            \begin{scope}[local bounding box=continuous_scope]
                \drawContinuousNeuronInner{cont. LN neuron}{0};
            \end{scope}

            % draw lIF neuron
            \begin{scope}[shift={(4.5,0)},local bounding box=spiking_scope]
                \drawSpikingNeuronInner{spiking lIF neuron}{0};
            \end{scope}
        \end{scope}

        % connect neurons
        \draw[->,relation] (binary_scope.north) |- node[above=3mm, pos=0.75, align=center]{\textbf{discrete}\\multi-level} (digital_scope.west);
        \draw[->,relation] let \p1=(digital_scope.east), \p2=(sigmadelta_scope.west) in
            (\x1,\y1) -- node[above=3mm, pos=0.5, align=center]{\textbf{pdm}\\(clocked)} (\x2,\y1);
        \draw[->,relation] (binary_scope.south) |- node[below=3mm, pos=0.75, align=center]{\textbf{continuous}\\multi-level} (continuous_scope.west);
        \draw[->,relation] let \p1=(continuous_scope.east), \p2=(spiking_scope.west) in
            (\x1,\y1) -- node[below=3mm, pos=0.5, align=center]{\textbf{spiking}\\(async.)} (\x2,\y1);
    
    \end{tikzpicture}
}






% draw adaptive neuron
\makeatletter
\newcommand{\drawPlasticNeuronInner}[3][\@nil]{
    \def\optarg{#1}
    \def\drawarrow{#3}
    %draw first neuron
    \node[operation]  (sum) at (0,0) {+};
    \node[weight, label={below:{$w_{1}$}}] (weight1) at ($(sum)+(-1,1.1)$) {$\times$};
    \node[weight, label={below:{$w_{2}$}}]  (weight2) at ($(sum)+(-1,0)$) {$\times$};
    \node[] (dots) at ($(sum)+(-1,-1.1)$) {\vdots};
    \draw node[block, label=above:{$x(t)$}] (filter) at ($(sum)+(1,0)$) {$\frac{\alpha}{s+\alpha}$};
    \draw node[block, linear=12pt] (suffstat1) at ($(filter)+(1,-1)$) {};
    \draw node[block, quadratic=12pt] (suffstat2) at ($(filter)+(1,-2)$) {};
    \draw node[block,label={above:{$\chi_1$}}] (suffstatfilt1) at ($(suffstat1)+(1,0)$) {$\frac{\beta}{s+\beta}$};
    \draw node[block,label={below:{$\chi_2$}}] (suffstatfilt2) at ($(suffstat2)+(1,0)$) {$\frac{\beta}{s+\beta}$};
    \draw node[block, exponential=12pt] (nonlinearity) at ($(filter)+(3,0)$) {};
    \coordinate (transmitter) at ($(nonlinearity)+(4em,0)$) ;

    % draw signals
    \ifnum\numexpr#3\relax=1{
        \draw[<-,decorate,continuous] (weight1) -- node[below=0.2]  (input1) {$s_1(t)$} ++(-4em,0);
        \draw[<-,decorate,continuous] (weight2) -- node[below=0.2]  (input2) {$s_2(t)$} ++(-4em,0);
    } \fi;
    
    \draw [->] (weight1) -| (sum);
    \draw [->] (weight2) -- (sum);
    \draw [] (sum) -- (filter) -- (nonlinearity);
    \draw [->] (filter) |-  (suffstat1); 
    \draw [->,>=|] (suffstat1) -- (suffstatfilt1) -| ($(nonlinearity.south west)+(5pt,-3pt)$);
    \draw [->] (filter) |-  (suffstat2);
    \draw [->,>=|] (suffstat2) -- (suffstatfilt2) -| ($(nonlinearity.south east)+(-5pt,-3pt)$);
    
    % draw background
    \begin{scope}[on background layer]
        \path [highlight=highlightColor_continuous] let \p1=(weight1.north west), \p2=(suffstatfilt2.south west), \p3=(nonlinearity.east) in 
        ($(\x1,\y1) + (-0.1,0.1)$) -- node[above]{#2} ($(\x3,\y1) + (0.1,0.1)$) -- ($(\x3,\y2)+(0.1,-0.5)$) -- ($(\x1,\y2)+(-0.1,-0.5)$) -- cycle;
    \end{scope}
    
    
    \ifnum\numexpr#3\relax=1{
        \draw[->,decorate,continuous] (nonlinearity) -- node[below=0.2]  (input2) {$y(t)$} (transmitter);
    } \else \ifnum\numexpr#3\relax>1 {            %draw second neuron
        \path (transmitter) ++(1.5,+1) node[weight, hidden, label={below:{$w_{3}$}}] (weight3) {$\times$}
        ++(0,-1) node[weight, label={below:{$w_{4}$}}]  (weight4) {$\times$}
        ++(0,-1) node[]  (dots2) {\vdots};
        \node[operation, hidden, right=of weight4, label={below:$z(t)$}]  (sum2) {+};
        \node[right=of sum2] {...};

        % draw background
        \begin{scope}[on background layer]
                \path [highlight=highlightColor_continuous, right color=white] let \p1=(weight3.north west), \p2=(dots2.south west), \p3=(weight3.west), \p4=(sum2.east) in 
            ($(\x3,\y1) + (-0.1,0.1)$) -- node[above]{#1} ($(\x4,\y1) + (0.2,0.1)$) -- ($(\x4,\y2)+(0.2,-0.1)$) -- ($(\x3,\y2)+(-0.1,-0.1)$) -- cycle;
        \end{scope}

        \draw[->,decorate,continuous] (nonlinearity) -- node[below=0.2]  (input2) {$y(t)$} (weight4);
        \draw [->, hidden] (weight3) -| (sum2);
        \draw (weight4) -- (sum2);
    } \fi \fi;
}
\makeatother
\newcommand{\drawPlasticNeuron}[3][1]{
    \begin{tikzpicture}
        \drawPlasticNeuronInner[#1]{#2}{#3};
    \end{tikzpicture}
}


% draw filter neuron
\makeatletter
\newcommand{\drawFilterNeuronInner}[5][\@nil]{
    \def\optarg{#1}
    \def\drawarrow{#3}
    \def\taps{#4}
    %draw first neuron

    \coordinate (origin) at (0,0);
    \draw node[block, relu=12pt] (nonlinearity) at ($(origin)+(2,0)$) {};
    \coordinate (transmitter) at ($(nonlinearity)+(4em,0)$);
    
    \def\offset{(origin)}
    \def\nextnode{(nonlinearity)}
    \def\numtaps{0}
    \foreach \tap[count=\i] in \taps{
        \node[operation]  (sum-\i) at \offset {+};
        \node[block, label=below:{$x_\i(t)$}]  (filter-\i) at ($(sum-\i)+(1,0)$) {\tap};
        
        \ifnum\i>1
            \fill (sum-\i.center)++(-0.25,1.5) circle[radius=1pt];
            \fill (sum-\i.center)++(0.25,1) circle[radius=1pt];
        \fi

        \draw[->] (sum-\i.center)++(-0.25,1.5) -- node[left]{$w_{\i,1}$} ++(0,-0.4cm) arc(90:270:1mm) -- ($(sum-\i.center)+(-0.25,0.5)$) -- (sum-\i);
        \draw[->] (sum-\i.center)++(0.25,1) -- ++(0,-0.1cm) -- node[right]{$w_{\i,2}$} ($(sum-\i.center)+( 0.25,0.5)$) -- (sum-\i);
        \draw (sum-\i) -- (filter-\i);
        \draw[->] (filter-\i) -- \nextnode;
        
        \xdef\offset{($(sum-\i)-(2,0)$)}
        \xdef\nextnode{(sum-\i)}
        \xdef\numtaps{\i}
    }
    \path \nextnode + (-1,0) node(tapdots) {$\cdots$};
    
    \path (tapdots) ++(-0.5,1) coordinate(pin-2) ++(0,0.5) coordinate(pin-1);
    \draw (pin-1) -- (-0.25,1.5);
    \draw (pin-2) -- ( 0.25,1.0);
    
    
    % draw signals
    \ifnum\numexpr#3\relax=1{
        \draw[<-,decorate,continuous] (pin-1) -- node[above]  (input1) {$s_1(t)$} ++(-4em,0);
        \draw[<-,decorate,continuous] (pin-2) -- node[below]  (input2) {$s_2(t)$} ++(-4em,0);
        \node[rotate=90] (inputdots) at ($(input2) + (0,-0.5)$) {$\dots$};
    } \fi;
    
    \ifnum\numexpr#5\relax=0{
        \coordinate (lower_right) at ($(nonlinearity.east)+(0,-1)$);
    } \else {
        \coordinate (lower_right) at ($(nonlinearity.east)+(0.1,-1.25)$);
        \draw[->,decorate,continuous] (nonlinearity) -- node[below=0.2]  (input2) {$y(t)$} (transmitter);
        
        \foreach \tap[count=\i] in \taps {
            \draw[->] (sum-\i.center)++(0,-1) -- node[left]{$v_\i$} (sum-\i);
            \ifnum\i<\numtaps
                \fill (sum-\i.center)++(0,-1) circle[radius=1pt];
            \fi
        };
        
        \ifnum\numexpr#5\relax=1 {
            \draw (nonlinearity.east)++(0.1,0) |- ($(sum-\numtaps)+(0,-1)$);
        } \else {
            \draw (filter-1.east)++(0.1,0) |- ($(sum-\numtaps)+(0,-1)$);
        }\fi;
     } \fi ;
    
    \coordinate (upper_left) at ($(pin-1)+(0,0.2)$);
    % draw background
    \begin{scope}[on background layer]
        \path [highlight=highlightColor_filter] let \p1=(upper_left), \p2=(lower_right) in 
        ($(\x1,\y1) + (-0.1,0.1)$) -- node[above]{#2} ($(\x2,\y1) + (0.1,0.1)$) -- ($(\x2,\y2)+(0.1,-0.1)$) -- ($(\x1,\y2)+(-0.1,-0.1)$) -- cycle;
    \end{scope}
    
    
    \ifnum\numexpr#3\relax=1{
        \draw[->,decorate,continuous] (nonlinearity) -- node[below=0.2]  (input2) {$y(t)$} (transmitter);
    } \fi;

}
\makeatother
\newcommand{\drawFilterNeuron}[5][1]{
    \begin{tikzpicture}
        \drawFilterNeuronInner[#1]{#2}{#3}{#4}{#5};
    \end{tikzpicture}
}


% draw multi-filter neuron
\makeatletter
\newcommand{\drawMultiFilterNeuronInner}[5][\@nil]{
    \def\optarg{#1}
    \def\drawarrow{#3}
    \def\taps{#4}
    %draw first neuron

    \coordinate (origin) at (0,0);
    \draw node[block, relu=12pt] (nonlinearity) at ($(origin)+(2,0)$) {};
    \node[operation]  (sum) at ($(nonlinearity)+(-1,0)$) {+};
    \coordinate (transmitter) at ($(nonlinearity)+(4em,0)$);

    \draw[->] (sum) -- (nonlinearity);
   
    \def\offset{($(sum)+(-1,0)$)}
    \def\nextnode{(nonlinearity)}
    \def\numtaps{0}
    \foreach \tap[count=\i] in \taps{
        \node[block, label=below:{$x_\i(t)$}, anchor=east]  (filter-\i) at \offset {\tap};
        \coordinate  (pin-\i) at ($(filter-\i)+(-1,0)$);

        % draw signals
        \ifnum\numexpr#3\relax=1{
            \draw[decorate,continuous] (pin-\i) -- node[below]  (input1) {$s_\i(t)$} ++(-4em,0);
        } \fi;

        \draw[->] (pin-\i) -- (filter-\i);
        \draw[->] (filter-\i.east) -- ++(0.25,0) -- (sum);
        
        \xdef\offset{($\offset+(0,-1.25)$)}
        \xdef\numtaps{\i}
    };

    \node (fdots) at ($(filter-\numtaps)+(0,-1.25)$) {\vdots};
    \ifnum\numexpr#3\relax=1{
        \node (idots) at ($(filter-\numtaps.east)+(-1,-1.25)+(-4em,0)$) {\vdots};
    } \fi;

    \coordinate (upper_left) at ($(pin-1)+(0,0.5)$);
    \coordinate (right) at (nonlinearity.east);
    \coordinate (lower) at ($(fdots)+(0,-0.5)$);

    % draw background
    \begin{scope}[on background layer]
        \path [highlight=highlightColor_filter] let \p1=(upper_left), \p2=(right), \p3=(lower) in 
        ($(\x1,\y1) + (-0.1,0.1)$) -- node[above]{#2} ($(\x2,\y1) + (0.1,0.1)$) -- ($(\x2,\y3)+(0.1,-0.1)$) -- ($(\x1,\y3)+(-0.1,-0.1)$) -- cycle;
    \end{scope}
    
    
    \ifnum\numexpr#3\relax=1{
        \draw[->,decorate,continuous] (nonlinearity) -- node[below=0.2]  (input2) {$y(t)$} (transmitter);
    } \fi;

}
\makeatother
\newcommand{\drawMultiFilterNeuron}[5][1]{
    \begin{tikzpicture}
        \drawMultiFilterNeuronInner[#1]{#2}{#3}{#4}{#5};
    \end{tikzpicture}
}

\newcommand{\drawBallStickNeuronInner}[2]{
    \coordinate (origin) at (0,0);
    \coordinate (soma) at ($(origin)+(2,0)$);
    \coordinate (transmitter) at ($(soma)+(4em,0)$);
    
    \def\numtaps{#1}
    \foreach \i in {1,...,\numtaps}{
        \coordinate (sum-\i) at ($(soma)+(-\i*2,0)$);
        
        \ifnum\i>1
            \fill (sum-\i.center)++(-0.25,1.5) circle[radius=1pt];
            \fill (sum-\i.center)++(0.25,1) circle[radius=1pt];
        \fi

        \draw[->] (sum-\i.center)++(-0.25,1.5) -- node[left]{$w_{\i,1}$} ++(0,-0.4cm) arc(90:270:1mm) -- ($(sum-\i.center)+(-0.25,0.25)$);
        \draw[->] (sum-\i.center)++(0.25,1) -- ++(0,-0.1cm) -- node[right]{$w_{\i,2}$} ($(sum-\i.center)+( 0.25,0.25)$);
    }
    
    \path (sum-\numtaps) ++(-1.5,1) coordinate(pin-2) ++(0,0.5) coordinate(pin-1);
    \draw (pin-1) -- (-0.25,1.5);
    \draw (pin-2) -- ( 0.25,1.0);
    
    % draw ball-stick
    \draw (soma)++(-0.5*0.707106781,0)++(225:0.5*0.707106781) arc (225:495:0.5*0.707106781) -- ++(-\numtaps*2,0) |- cycle;
    
    % % draw signals
    \draw[<-,decorate,continuous] (pin-1) -- node[above]  (input1) {$s_1(t)$} ++(-4em,0);
    \draw[<-,decorate,continuous] (pin-2) -- node[below]  (input2) {$s_2(t)$} ++(-4em,0);
    \node[rotate=90] (inputdots) at ($(input2) + (0,-0.5)$) {$\dots$};
    \draw[->,decorate,continuous] (soma) -- node[below=0.2]  (input2) {$y(t)$} (transmitter);
    
    \coordinate (lower_right) at ($(soma.east)+(0.1,-0.5)$);
    \coordinate (upper_left) at ($(pin-1)+(0,0.2)$);
    % draw background
    \begin{scope}[on background layer]
        \path [highlight=highlightColor_filter] let \p1=(upper_left), \p2=(lower_right) in 
        ($(\x1,\y1) + (-0.1,0.1)$) -- node[above]{#2} ($(\x2,\y1) + (0.1,0.1)$) -- ($(\x2,\y2)+(0.1,-0.1)$) -- ($(\x1,\y2)+(-0.1,-0.1)$) -- cycle;
    \end{scope}

}
\newcommand{\drawBallStickNeuron}[2]{
    \begin{tikzpicture}
        \drawBallStickNeuronInner{#1}{#2};
    \end{tikzpicture}
}


\drawDigitalNeuron{Digital Neuron}{1}
\pagebreak

\drawBinaryNeuron{Binary Neuron}{1}
\pagebreak

\drawContinuousNeuron{Continuous Neuron}{1}
\pagebreak

\drawSigmaDeltaModulator{Sigma-Delta-Modulator}{1}
\pagebreak

\drawSpikingNeuron{Spiking Neuron}{1}
\pagebreak

\drawSpikingNeuronSingleInput{Spiking Neuron}{1}
\pagebreak

\drawRelationships
\pagebreak

\drawPlasticNeuron{Plastic Neuron}{1}
\pagebreak

\drawFilterNeuron{Gamma neuron with linear feedback}{1}{{$\frac{\alpha}{s+\alpha}$},{$\frac{\alpha}{s+\alpha}$},{$\frac{\alpha}{s+\alpha}$}}{2}
\pagebreak

\drawMultiFilterNeuron{Filter-Nonlinear neuron}{1}{{$\mathcal{K}_1$},{$\mathcal{K}_2$},{$\mathcal{K}_3$}}{0}
\pagebreak

\drawBallStickNeuron{3}{Ball \& Stick Neuron}
\pagebreak
